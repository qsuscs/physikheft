% !TeX spellcheck = de_DE
% !TeX encoding = UTF-8
\documentclass[a4paper]{scrartcl}

\usepackage[utf8]{inputenc}
\usepackage[ngerman]{babel}
\usepackage[T1]{fontenc}
\usepackage{amsfonts}
%\usepackage{amsmath}


\begin{document}

\subsubsection{"`Ausdrücke"' für die magnetische Flussdichte}
\(B:=\frac{F}{I \cdot l}\) \(\leftarrow\)Definition, also eine willkürliche
(aber natürlich sinnvolle) Festlegung\\ %TODO
\(B=\frac{F}{q \cdot v}\)aus der Definition abgeleitet für bewegte Ladungen\\
\(B=\frac{E}{v}\)aus der Gleichgewichtsbedingung el. und magn. Kraft
(\(F_{el}=F_{magn} \Longleftrightarrow qE=q \cdot v \cdot B\))\\
\(B=\mu_0 \cdot \frac{I}{2 \pi r}\)Flussdichte eines geraden stromdurchflossenen
Leiters\\
\(B=\mu_0 \cdot I \cdot \frac{n}{l}\)Flussdichte im innern einer
"`langen"' Spule

\subsubsection{Magnetfeld einer Spule}
\$BILD\\
\(l\)= Spulenlänge\\\(n\)= Windungszahl\\\(I\)= Spulenstrom\\
Innerhalb der Spule ist das Magnetfeld näherungsweise homogen, außerhalb
näherungsweise Null, wenn es sich um eine "`\emph{lange Spule}"' handelt, d.h.
die Länge viel größer als ihr Durchmesser ist.

Im Prinzip lässt sich der Ausdruck für B aus dem für den geraden Leiter
herleiten (s. Buch S. 248f). Es gilt \(B=\mu_0 I \frac{n}{e}\). \(\frac{n}{e}\)
kann als Windungsdichte interpretiert werden.

\emph{Beispiel:} \(l=40\mathrm{cm}=0,4\mathrm{m}\); \(n=30\);
\(I=2\mathrm{A}\)\\
\(B=\mu_0 \cdot I \cdot \frac{n}{e}=4 \pi \cdot 10^{-7}
\frac{\mathrm{Vs}}{\mathrm{Am}} \cdot 2\mathrm{A} \cdot
\frac{30}{0,4\mathrm{m}}=\frac{4 \pi \cdot 30 \cdot 2}{0,4} \cdot 10^{-7}
\frac{\mathrm{Vs}}{\mathrm{Am}}= 1900 \cdot 10^{-7} T = 1,9 \cdot 10^{-4}
\mathrm{T} = 0,19 \mathrm{mT}\)

\subsubsection{Induktion (qualitativ)}
\emph{Induktion} bedeutet, dass mit Hilfe eines Magnetfeldes in einem Leiter
ein Strom erzeugt wird bzw. zwischen den Leiterenden eine Spannung messbar wird.

\paragraph{Mögliche Situationen}
\begin{itemize}
	\item Der Leiter bzw. die Spule wird quer zum Magnetfeld bewegt (bei Spulen
	insbesondere auch Rotation im Feld\(\rightarrow\)Generator). Die vom Leiter
	mitgeführten  Elektronen erfahren eine Lorentzkraft \emph{\underline{in
	Richtung}} des Leiters\(\Rightarrow\)Strom.
	\item Innerhalb einer Spule (bzw. in der Nähe eines Leiters)
	\underline{ändert}
sich das Magnetfeld. Hier kann die Induktion \emph{nicht} mit der Lorentzkraft
erklärt werden.
\end{itemize}

\subsubsection{Induktion (quantitativ)}
Versuch/Idee: In eine stromdurchflossene Spule (\emph{Feldspule}) wird eine
kleine Spule (\emph{Induktionsspule}) gebracht.
\$BILD
Die Feldspule ist an einen Sägezahngenerator angeschlossen, der die Stromstärke
linear an-- und absteigen lässt, wodurch die magnetische Flussdichte in der
Spule entsprechend an-- und absteigt: \(B = \mu_0 \cdot I \cdot \frac{n}{l}\)
\$BILD (\(B\)-\(t\)-Diagramm der Sägezahnspannung)
Die Induktionsspannung hängt ab von:
\begin{itemize}
	%\item Änderungsrate von \(B(t) \rightarrow \dot{B}(t)\)
	\item Induktionsspannung hängt ab von\(\dot{B}(t)\)(Magnetfeldänderung)
		(\(\dot{B}(t)\)hängt ab von \(I_{max}\)und\(f\))
	\item\(U_{ind}\)hängt ab von der Querschnittsfläche\(A\)der Ind.--Spule
	\item\(U_{ind}\)hängt ab von der Windungszahl der Ind.--Spule
\end{itemize}

\paragraph{Messungen}
\(n_{ind} = 1000\); %}
%$A_{ind} = 28\mathrm{cm^2}$; %}
\(I_{max} = 1\mathrm{A}\); %} U_{ind} = 0,01V
\(f=0.15\mathrm{Hz}\)%}
\(n_{Feld} = 665\)%} % Versuch 2: f=1Hz %U_ind=0,02V
\(\Rightarrow \dot{B}~U_{ind}\)%TODO Tilde
%TODO 3 wie 1, aber n_ind=2000 → U_ind=0.02V → U_ind~n
%TODO 4 wie 3, A_ind=14cm^2, U_ind = 0,01V U_ind ~ A
Insgesamt ergibt sich also \(U_ind ~ n \cdot A \cdot \dot{B}\). Wie groß ist der
Proportionalitätsfaktor?
\begin{enumerate}
	\item Einheit des Prop.--faktors:\([n]=1\)\([A]=\mathrm{cm^2}\)
	\([B]=\mathrm{T}=
		\frac{\mathrm{Vs}}{\mathrm{m^2}}\)
		\([\dot{B}]=\frac{\mathrm{V}}{\mathrm{m^2}}\)
		
		\(\Rightarrow [n \cdot A \cdot \dot{B}]=V=[U_{ind}]\)
		selbe Einheit, d.h. der Prop.--faktor ist dimensionslos (d.h. hat keine
		Einheit)
	\item Wir zeigen anhand der Messwerte, dass der Prop.--faktor 1 ist,
		dass also gilt: \(U_{ind} = n \cdot A \cdot \dot{B}\)
		%TODO n = 1000, A=28cm^2=0,0028m^2, I_max=1, B_max=\mu_0*I_max*n/l =
		%4\pi *10^-7 Vs/Am *1A 665/0,5m, l=0,5mA, B'=B_max/0,5s
		%TODO U_ind = n_ind*A_ind*B'_Feld =
		%1000*0,0028m^2*(4pi*10^-7Vs/Am*1A*665)/0,5m*0,5s
		%TODO = 2,8*4pi*665/0,5^2 * 10^3*10^-3*10^-7 * VsAm^2/ms
\end{enumerate}

In weiteren Versuchen stellt man fest, dass Verallgemeinert ausgedrückt die
\emph{Änderung} der von dem magnetischen Fluss durchsetzten Spulenfläche
eine Induktionsspannung bewirkt. Dies kann hervorgerufen werden durch
\begin{enumerate}
	\item Zusammendrücken/Auseinanderziehen der Spule im Feld \$BILD %TODO
	\item Herausziehen/Hineinschieben d. Sp. aus dem/in das Feld \$BILD %TODO
	\item Rotation der Spule im Feld
\end{enumerate}
Es gilt hier: \(U_{ind} = n_{ind} \cdot \dot{A}_{ind} \cdot B_{Feld}\), wobei
\([\dot{A}]=\frac{\mathrm{m^2}}{\mathrm{s}}\) die
"`\emph{Flächenänderungsrate}"'. %TODO

Wir legen fest: Der \emph{magnetische Fluss} durch eine Fläche ist (bei
homogenem Feld) das Produkt aus mgn. Flussdichte mal betrachteter Fläche:
\(\Phi = A \cdot B\) (so ist die Flussdichte \(B=\frac{\Phi}{A}\)) %TODO
%Fluss frac Fläche
Damit lässt sich das Induktionsgesetz ganz kompakt schreiben:
\(U_{ind} = n \cdot \dot{\Phi}\)(Induktionsspannung ist n mal Änderungsrate
des magnetischen Flusses)

(\(\dot{\Phi}=\dot{A \cdot B} = \dot{A} \cdot B + A \cdot \dot{B}\)
(Produktregel))

\subsubsection{Differentialgleichung}
Fall: Entladung/Ausschalten; Schaltung: \(C\) oder \(L\) parallel zu \(R\)
geschalten, also \(U_C\) bzw. \(U_L\) und \(U_R\)
\\ \begin{tabular}{rl|l}
& Kondensator						& Spule \\
"`Konstanten"'&\(R\);\(C\)			&\(R\);\(L\)\\
Zeitabhängige Größen &\(Q(t)=\mathcal{Q}\);\(U(t)=\mathcal{U}\);
	\(I(t)=\mathcal{I}\)				&\(U(t)=\mathcal{U}\);
	\(I(t)=\mathcal{I}\)\\
&\(U_C + U_R = 0\)&\(U_L + U_R = 0\)\\
&\(\frac{1}{C} \mathcal{Q} + R \cdot \mathcal{I} = 0\)&
	\(L \cdot \dot{\mathcal{I}} + R \cdot \mathcal{I}\)= 0 \\
&\(\frac{1}{C} \mathcal{Q} + R \dot{\mathcal{Q}} = 0\)&
	\(\dot{\mathcal{I}} + \frac{R}{L} \cdot \mathcal{I} = 0\)\\
&\(\frac{1}{RC} \mathcal{Q} + \dot{\mathcal{Q}} = 0\)&
	\(\dot{\mathcal{I}} = -\frac{R}{L} \cdot \mathcal{I}\)\\
&\(\dot{\mathcal{Q}} = - \frac{1}{RC} \mathcal{Q}\)& \\
\hline % Einmal quer drüber "`Lösung der DGL"'
&\(\mathcal{Q} = Q(t) = Q_0 \cdot e^{-\frac{1}{RC} \cdot t}\)&
	\(\mathcal{I} = I(t) = I_0 \cdot e^{-\frac{R}{L} \cdot t}\)\\
& entsprechend auch für\(U(t)\)und\(I(t)\)& entsprechend auch für\(U(t)\)\\
\emph{"`Probe"':} &\(\dot{Q} = Q_0 \cdot e^{-\frac{1}{RC} \cdot t}\)&
	\(\dot{I}(t) = I_0 \cdot e^{\frac{R}{L} \cdot t} \cdot -\frac{R}{L}\)\\
& Q/I/U-t-Diagramm & I/U-t-Diagramm \\
\end{tabular}
\paragraph{Einheitenbetrachtung:} \([\frac{1}{RC}] = \mathrm{\frac{1}{\Omega
\cdot F} = \frac{1}{\frac{V}{A} \cdot \frac{C}{V}} = \frac{A}{C} = \frac{A}{As}
= \frac{1}{s}}\);
\([\frac{R}{L}] = \mathrm{\frac{\Omega}{H} = \frac{\frac{V}{A}}{\frac{Vs}{Am}
\cdot \frac{m^2}{m}} = \frac{V \cdot Am \cdot m}{Vs \cdot A \cdot m^2} =
\frac{1}{s}}\)


\section{Mechanik -- Schwingungen und Wellen}
\subsection{Die Schwingung einer Schraubenfeder}
\paragraph{Physikalische Größen:} \begin{itemize}
	\item \emph{Frequenz} \(f\) \(\mathrm{Hz\,(Hertz)} = \frac{1}{\mathrm{s}}\)
	\item \emph{Schwingungsdauer} \(T\) \(\mathrm{s}\) \(f = \frac{1}{T}\)
	\item \emph{Amplitude} \(\hat{y}\) \(\mathrm{m}\)
	\item \emph{"`Kreisfrequenz"'} \(\omega = 2 \pi f\) \(\frac{1}{\mathrm{s}}\)
	\item \emph{Auslenkung} \(y(t)\) \(\mathrm{m}\)
	\item \emph{Federkonstante} \(D\) \(\frac{\mathrm{N}}{\mathrm{m}}\)
	\item \emph{Masse} \(m\) \(\mathrm{kg}\)
\end{itemize}
%TODO Bild: Oben "`Decke"', daran befestigt eine Feder, darauf ein grüner Pfeil
%F_D = D*s (Dehnung der Feder), an der Feder eine Masse, davon nach unten ein
%roter Pfeil F_G = m*g, neben der Masse in orange "`Gleichgewichts- oder
%Ruhelage"'
In der Gleichgewichtslage gilt Kräftegleichheit: \(F_D = F_G \rightarrow m \cdot
g = D \cdot s_0\)

\subsubsection{Versuch: Masse an einer senkrecht aufgehängten Feder}
\paragraph{Aufbau}
An einem Stativ wird eine Schraubenfeder senkrecht befestigt. An ihrem unteren
Ende befestigt man eine Schnur, an deren Ende wiederum eine Masse von 200
Gramm. Die Schnur berührt dabei das sich frei drehende Rad eines Messgerätes.
Auf einem Bildschirm wird mithilfe des Messgerätes die Auslenkung der Feder
angezeigt und grafisch (\(y(t)\)) dargestellt.
\paragraph{Beobachtung}
Wenn man die Feder in Schwingung versetzt, sieht der Funktionsgraph der
Auslenkung über der Zeit wie eine Sinusfunktion aus.

\subsubsection{Beschreibung der Schwingung durch eine Differentialgleichung}
Auftretende Kräfte: Außerhalb der Gleichgewichtslage\(s_0\)ist die Federkraft
\(F_D\)größer oder kleiner als \(F_G = m \cdot g\), die konstant ist. Diese
resultierende Kraft ist die sogen. \emph{rücktreibende Kraft}, die immer in
richtung Ggl. zeigt. Es gilt: \(F_R = F_D + F_G = -Ds + mg = -D(y+s_0) + mg =
-Dy-Ds_0 + mg = -Dy\) \(F_R = -Dy\)%TODO Kasten rum

\(F_R\)ist proportional zur Auslenkung aus der Gleichgewichtslage\(y=0\)!
(\emph{lineares Kraftgesetz}).
\(F_R\)ist damit die die Masse \(m\) beschleunigende Kraft \(F=m \cdot a\).
Also: \(F = F_R \rightarrow m \cdot a(t) = -D \cdot y(t)\). Da \(\dot{y}(t) =
v(t)\) und \(\dot{v}(t) = a(t)\) gilt \(m \cdot \ddot{y}(t) = -D \cdot y(t)\).
Geteilt durch \(m\) ergibt sich \(\ddot{y}(t) = -\frac{D}{m} \cdot y(t)\)

Diese DGL hat die Struktur \(f''(x) = -k \cdot f(x)\). Diese wird gelöst durch
die trigonometrischen Funktionen\(\sin(x)\)/\(\cos(x)\) bzw. genauer \(f(x) =
\sin(\sqrt{k}x)\). Denn \(f'(x) = \sqrt{k} \cos(\sqrt{k}x)\) und \(f''(x) =
-\sqrt{k}^2 \sin(\sqrt{k}x) = -k \sin(\sqrt{k}x)\)

Lösung: \(y(t) = \hat{y} \cdot \sin(\sqrt{\frac{D}{m}}t)\)(\([\hat{y}] =
\mathrm{m}\);\([\sqrt{\frac{D}{m}}][t] = \mathrm{s} \rightarrow
[\sqrt{\frac{D}{m}}] = \frac{1}{\mathrm{s}}\)
\\
\([\frac{D}{m}] = \frac{\mathrm{N}}{\mathrm{m \cdot kg}} =
\frac{1}{\mathrm{s^2}} \rightarrow [\sqrt{\frac{D}{m}}] = \frac{1}{\mathrm{s}}\)
\\
\(v(t) = \dot{y}(t) = \sqrt{\frac{D}{m}} \cdot \hat{y} \cdot
\cos(\sqrt{\frac{D}{m}}t)\); \(\sqrt{\frac{D}{m}} \cdot \hat{y} = \hat{v}\)
\\
\(a(t) = \ddot{y}(t) = -\sqrt{\frac{D}{m}}^2 \cdot \hat{y} \cdot
\sin(\frac{D}{m} \cdot t)\); \(-\sqrt{\frac{D}{m}}^2 \cdot \hat{y} = \hat{a}\),
\(\hat{y} \cdot \sin(\frac{D}{m} \cdot t) = y(t)\)

Zusammenhang zwischen \(y(t) = \hat{y} \sin(\sqrt{\frac{D}{m}} t)\) mit
\(f\)bzw. \(T\)der Schwingung: Periodendauer/länge: \(\sqrt{\frac{D}{m}} \cdot
T = 2 \pi\), d.h.\(T=\frac{2 \pi}{\sqrt{\frac{D}{m}}} = 2 \pi
\sqrt{\frac{m}{D}}\) \(f=2 \pi f = \sqrt{\frac{D}{m}} = \omega\)

\subsection{Überblick: \emph{Harmonischer Oszillator} (HO)}
Ein schwingungsfähiges System liegt dann vor, wenn ein Körper/eine Masse durch
eine \emph{rücktreibende Kraft} in Richtung einer \emph{Ruhelage} beschleunigt
wird. Ist diese rücktreibende Kraft "`\emph{linear}"' (\(F_R=-Dy\)), d.h.
proportional zur Auslenkung aus der Ruhelage, so ergibt sich eine
\emph{harmonische} Schwingung, d.h. die Auslenkungsfunktion \(y(t)\) ist
\emph{sinusförmig}. Die Größe \(D\) ist beim Federpendel die
\emph{Federkonstante}, bei anderen Systemen wird sie allg. als
\emph{Richtgröße} bezeichnet. Für den HO gilt allgemein:\\
Kräftegleichung \(F_a=F_R \rightarrow m \cdot \ddot{y}=-D\cdot y(t)\) DGL mit:
\(y(t) = \hat{y} \cdot \sin(\omega t);\, v(t) = \dot{y}(t) = \omega \cdot
\hat{y} \cos(\omega t),\, (\omega \cdot \hat{y}=\hat{v});\, a(t)=\ddot{y}(t) =
- \omega^2 \hat{y} \sin(\omega t),\, (\omega^2 \hat{y}=\hat{a})\) und
\(\omega^2=\frac{D}{m}\) bzw. \(\omega=\sqrt{\frac{D}{m}}\) und \(\omega = 2
\pi f\) und \(T = \frac{1}{f}\)

\subsection{Energie des HO}
Im HO auftretende Energieformen: \(\mathcal{E}_{kin}\) und \(\mathcal{E}_{pot}\)
(enthält Spannenergie der Feder und Lageenergie der Masse) mit
\(\mathcal{E}_{kin} = \frac{1}{2}mv^2\)und\(\mathcal{E}_{spn} =
\mathcal{E}_{pot} = \frac{1}{2}Dy^2 = \int F(s) ds\). Mit\(v(t) = \hat{v}
\cos(\omega t)\)und\(y(t) = \hat{y} \cdot \sin(\omega t)\) ergibt sich
\(\mathcal{E}_{kin}(t) = \frac{1}{2} m (\hat{v} \cos(\omega t))^2 = \frac{1}{2}
m \hat{v}^2 \cos^2(\omega t)\) und \(\mathcal{E}_{pot}(t) = \frac{1}{2} D
(\hat{y} \sin(\omega t))^2 = \frac{1}{2} D \hat{y}^2 \sin^2(\omega t)\)

\subsubsection{Betrachtung der Energien als Funktion von \(t\) bzw. \(y\)}
Mit \(\mathcal{E}_{pot} = \frac{1}{2}Dy^2\) und \(\mathcal{E}_{kin} =
\frac{1}{2}mv^2\) erhält man entweder:
\begin{enumerate}
	\item mit \(y=y(t)\) und \(v=v(t)\) \(\mathcal{E}_{pot}(t)\) bzw.
		\(\mathcal{E}_{kin}(t)\)
	\item mit \(y\) direkt \(\mathcal{E}_{pot}(y)\) und  mit \(\hat{v}=\omega
		\hat{y}\) \(\mathcal{E}_{kin}(y)\)
\end{enumerate}
Konkret:
\begin{enumerate}
	\item \(\mathcal{E}_{pot}(t) = \frac{1}{2} D (y(t))^2 = \frac{1}{2} D
		(\hat{y} \sin(\omega t))^2 = \frac{1}{2} D \hat{y}^2 \sin^2(\omega t)\)
		\\
		\(\mathcal{E}_{kin}(t) = \frac{1}{2} m (v(t))^2 = \frac{1}{2} m
		(\omega \hat{y} \cos(\omega t))^2 = \frac{1}{2} m \omega^2 \hat{y}^2
		\cos^2(\omega t)\)
		\(\longrightarrow \mathcal{E}_{ges} = \frac{1}{2} D \hat{y} =
		\frac{1}{2} m \omega^2 \hat{y}^2\) (es gilt: \(\sin^2(\omega t) +
		\cos^2(\omega t) = 1\))
		\\
		außerdem: \(\sin^2(\omega t) = \frac{1}{2} \sin^2 (2 \omega t) + 0,5\)
	\item \(\mathcal{E}_{pot}(y) = \frac{1}{2} D y^2(t) = \frac{1}{2}D y^2\)
		\\
		\( \mathcal{E}_{kin}(y) = \frac{1}{2} m v^2(t) =  \frac{1}{2} m (\omega
		\cdot \hat{y} \cos(\omega t))^2 = \frac{1}{2} m \omega^2 \hat{y}^2
		\cos^2(\omega t) = \frac{1}{2} D \hat{y}^2 \cos^2(\omega t) =
		\frac{1}{2} D \hat{y}^2 ( 1 - \sin^2(\omega t) ) = \frac{1}{2} D
		\hat{y}^2 - \frac{1}{2} D \hat{y}^2 \sin^2(\omega t) = \frac{1}{2} D
		\hat{y}^2 - \frac{1}{2} D y^2 = \mathcal{E}_{ges} -
		\mathcal{E}_{pot} \); \( \cos^2(\omega t) = 1 - \sin^2(\omega t);\,
		\hat{y}^2 \sin^2(\omega t) = y^2 \)
\end{enumerate}
\emph{Frage:} Für welchen Wert von \(y\) ist \( \mathcal{E}_{kin} =
\mathcal{E}_{pot} \)? \( \frac{1}{2}Dy^2 = \frac{1}{2}D\hat{y}^2 - \frac{1}{2}
Dy^2 \Rightarrow y = \sqrt{\frac{1}{2}}\hat{y} \)

\subsection{Das Fadenpendel}
% Zeichnung #0a0e7267-fb46-40f4-80d2-c7605effecab
\begin{enumerate}
	\item Rücktreibende Kraft\\
		\( \frac{F_R}{F_G} = \sin(\alpha) \Rightarrow F_R = F_G \cdot
		\sin(\alpha);\, F_R = m \cdot g \cdot \sin(\alpha) \)
\end{enumerate}

\subsection{Mechanische Wellen}
Eine \emph{Welle} ist ein Phänomen, das auf einem System gekoppelter
Oszillatoren stattfindet: Eine Erregung des ersten Oszillators wird über die
Kopplung an die anderen Oszillatoren weitergereicht. Wir betrachten den
Spezialfall der \emph{linearen harmonischen Welle}, d.h. der Wellenträger ist
eindimensional, die einzelnen Oszillatoren schwingen harmonisch, die Anregung
des "`1. Oszillators"' erfolgt dauerhaft. Entlang der Ausbreitungsrichtung der
Welle wird Energie transportiert, die Oszillatoren selbst bleiben dabei an
ihrem Platz (kein Materietransport, nur die Schwingung wird weitergegeben). Die
Welle ist charakterisiert durch die Schwingungsdauer \(T\) des einzelnen
Oszillators, die sogenannte \emph{Phasengeschwindigkeit}, d.h. die
Ausbreitungsgeschwindigkeit der Erregung und die \emph{Wellenlänge}, d.h. der
kürzeste Abstand zweier Oszillatoren gleicher Phase. Eine Welle ist zweifach
periodisch: in der Zeit bei festem Ort: jeder einzelne Oszillator schwingt
gemäß \(y(t) = \hat{y} \sin(\omega t + \varphi)\) sowie zu fester Zeit
(\emph{Momentbild}) im Ort: Die Auslenkung aller Oszillatoren folgt \(y(x) =
\hat{y} \sin(kx)\).\\
Beschrieben wird diese Welle durch die \emph{Wellenfunktion} (im Buch
"`Wellengleichung"')

\paragraph{Herleitung:} Es gilt: \(\omega = 2 \pi f = 2 \pi \frac{1}{T}; v_{ph}
= \frac{\lambda}{T} \)

Der "`erste"' Oszillator (\(x=0\)) schwingt gemäß \(y(x;t) = y(0;t) = y(t) =
\hat{y} \sin(\omega t)\)\\
Ein beliebiger Oszillator an einer Stelle \(x_1\) übernimmt diese Schwingung
\underline{zeitverzögert} zum Zeitpunkt \(t-t_1\):\\
\(\Rightarrow y(x_1;t) = \hat{y} \sin(\omega (t-t_1))\)\, Nun bringen wir über
\(t_1\) \(\lambda\) und \(v_{ph}\) ins Spiel: es gilt \(v_{ph} =
\frac{x_1}{t_1} = \frac{\lambda}{T} \rightarrow t_1 = \frac{x_1}{\lambda} \cdot
T = \hat{y} \sin(\frac{2\pi}{T} )(t - \frac{x_1}{\lambda} \cdot T) = \hat{y}
\sin(2\pi (\frac{t}{T} - \frac{x_1}{\lambda})) \rightarrow\) \fbox{\(y(x;t) =
\hat{y} \sin(2\pi (\frac{t}{T} - \frac{x_1}{\lambda}))\)} (Wellenfunktion)

Diese Funktion berechnet zu vorgegebenen \(T\); \(\lambda\) und \(\hat{y}\) die
Auslenkung des Oszillators an beliebiger Stelle \(x\) zu beliebiger Zeit \(t\).
Hält man z.B. die \underline{Zeit} fest, so erhält man \(y(x)\), also den
Graphen aller Oszillatoren zu diesem Zeitpunkt (Momentbild). Betrachtet man
einen festen Ort, so erhält man y(t), also das Schwingungsbild des Oszillators
an diesem Ort.

\subsection{Überlagerung von Wellen -- Interferenz}
\paragraph{Begriffe:} \begin{itemize}
	\item \emph{Wellenberg/Wellental}: Orte, an denen der entsprechende
		Oszillator seine maximale Auslenkung hat (nach oben/unten).
	\item \emph{Interferenz}: Überlagerung von 2 oder mehr Wellen. Dies
		geschieht "`ungestört"', d.h. es addieren sich einfach die
		entsprechenden Auslenkungen.
	\item \emph{Konstruktive Interferenz}: Überlagerung von Wellen gleicher
		Phase, die Einzelamplituden addieren sich.
	\item \emph{Destruktive Interferenz}: Überlagerung von zwei Wellen
		entgegengesetzter Phase, bei gleicher Einzelamplitude erfolgt völlige
		Auslöschung
	\item \underline{Bemerkungen}: Wenn sich drei oder mehr Wellen destruktiv
		überlagern, ist die Situation komplizierter.\\
		In einem Wellenfeld gibt es feste Orte destruktiver Interferenz,
		sogenannte \emph{Knoten}
\end{itemize}

\subsubsection{Interferenz von Schallwellen}
Für \(f\); \(\lambda\) und \(c\) gilt: \(c=\lambda\cdot f\)\\
speziell für Schall in Luft ist \(c=340\mathrm{\frac{m}{s}}\), wir wählen
\(f=3400\mathrm{Hz} \Rightarrow \lambda=0,1\mathrm{m}\)
\paragraph{Versuch:} Interferenzen bei Schallwellen\\
Zeichung: 7ece6ca5-05e1-4a70-addd-77b94a185a9f, Teil 1 %TODO
\subparagraph{Durchführung:} Zwei Lautsprecher werden von einem
Frequenzgenerator gespeist (\(f=3400\mathrm{Hz}\)). Ein Mikrofon wird in
einer Linie parallel zur Verbindungsstrecke der Lautsprecher bewegt.
\subparagraph{Beobachtung:} Die Amplitude des Mikrofons ändert sich mehrfach,
von Maximum zu Minimum.
Zeichnung: 7ece6ca5-05e1-4a70-addd-77b94a185a9f, Teil 2\\
\(\Rightarrow\) Es gibt 5 \emph{Interferenzhyperbeln konstruktiver Interferenz}
(diese haben untereinander auf der Verbindungsstrecke zwischen \(E_1\) und
\(E_2\) jeweils den Abstand \(\frac{\lambda}{2}\))

\subsection{Stehende Wellen}
Überlagern sich auf einem Wellenträger zwei gegenläufige Wellen gleicher
Wellenlänge und Amplitude, so erhält man eine \emph{stehende Welle}:
An jeder Stelle haben die Anteile der rechts- bzw. linkslaufenden Welle eine
konstruktive Phasenbeziehung, sodass die Amplitude \underline{aller}
Oszillatoren konstant bleibt. An Stellen entgegengesetzter Phase ist die
resultierende Amplitude Null, man hat einen \emph{Knoten}, bei gleicher Phase
ist die Amplitude verdoppelt, man hat einen \emph{Bauch}. Eine stehende Welle
transportiert die Energie \underline{nicht}, sondern "`speichert"' sie
sozusagen.

Zeitlicher Verlauf: Hier sollte jetzt ein Bild sein. "`Das könnt ihr so
freihand zeichnen."' Hahahahahaha nein.

Man unterscheidet bei Wellenträgern \emph{"`feste"'} bzw. \emph{"`offene"'}
\emph{Enden}, die sich durch ihr Reflexionsverhalten der hin- und herlaufenden
Wellen unterscheiden:
\begin{itemize}
	\item[\underline{offenes Ende}:]  Reflexion "`mit Phasensprung"'
		\(\Rightarrow\) hin- und rücklaufende Welle überlagern sich zu einem
		Knoten 	(\emph{bezogen auf den Druck})
	\item[\underline{festes Ende}:] Reflexion "`ohne Phasensprung
		\(\Rightarrow\) hin- und rücklaufende Welle überlagern sich zu einem
		Bauch (\emph{Druck})
\end{itemize}
Bezogen auf die Teilchen\underline{bewegung} hat man beim offenen Ende einen
Bauch, bei geschlossenen einen Knoten. Mit diesen \emph{Randbedingungen} sind
folgende stehende Wellen möglich:\\
Wellenträger mit (Länge \(l\))\\
Haha Bilder schon wieder.\\
\(l = \frac{\lambda}{2} \Leftrightarrow \lambda = 2l\)\\
\(c = 340 \frac{\mathrm{m}}{\mathrm{s}}\)\\
Aus \(c = \lambda \cdot f\Rightarrow\)\\
Bsp.: \(f = \frac{c}{\lambda} = \frac{340 \frac{\mathrm{m}}{\mathrm{s}}}{2
\cdot 63 \mathrm{cm}} = \frac{340 \frac{\mathrm{m}}{\mathrm{s}}}{2 \cdot 0,63
\mathrm{m}} = 270 \mathrm{Hz}\)\\
Noch ein Bild.\\
\(l = \frac{\lambda}{4} \Leftrightarrow \lambda = 4l\)\\
\(f = \frac{c}{\lambda} = \frac{340 \frac{\mathrm{m}}{\mathrm{s}}}{4 \cdot 63
\mathrm{cm}} = 135 \mathrm{Hz}\)

\subsection{Überblick}
\begin{itemize}
	\item[\underline{Knoten} (destruktiv):] an den Punkten, bei denen der
		\emph{Gangunterschied} \(\Delta S = k \frac{\lambda}{2}\) mit \(k=1; 3;
		5; \ldots\)
	\item[\underline{"`Bauch"'} (konstruktiv):] für Gangunterschied \(\Delta S
		=  k \cdot \lambda\) mit \(k \in \mathbb{N}^0\)
\end{itemize}
Hyperbel (geometrische Definition): Menge aller Punkte \(A_i\), deren Abstände
zu zwei festen Punkten \(E_1\) und \(E_2\) konstante Differenz haben:
\(d=d_{22} - d_{21} = d_{12} - d_{11}\)

\subsection{Töne \& Klänge}
Eine schwingende Luftsäule bestimmter Länge \(\l\) kann verschiedene stehende
Wellen ausbilden. Diejenige mit der tiefsten Frequenz heißt
\emph{Grundschwingung}, diejenigen höherer Frequenzen sind die
\emph{Oberschwingungen}. Ein \emph{Klang} entsteht dadurch, dass sich
gleichzeitig Grund- und Oberschwingungen mit verschiedenen Amplituden
ausbilden. Das zugehörige Schwingungsbild ist dann auch \underline{nicht} mehr
sinusförmig, aber periodisch. Für die Frequenzen gilt: Bild *gähn*\\
D.h. für den Grundton ist \(f_0 = \frac{c}{2l}\), für die Obertöne gilt \(f_1 =
\frac{c}{2l} \cdot 2\qquad f_2 = \frac{c}{2l} \cdot 3\) usw., d.h. die
Obertonfrequenzen sind ganzzahlige Vielfache der Grundtonfrequenz.

\section{Elektromagnetische Wellen}
\subsection{Geometrische Optik}
Phänomene: Brechung; Reflexion (Totalreflexion; Brewster-Gesetz)\\
Versuch (Brechungsgesetz)
\subsubsection{Versuch}
\paragraph{Durchführung:} Wir variieren den \emph{Einfallswinkel} zwischen
	\(0^\circ\) und \(90^\circ\) (Winkel zum \textbf{Lot!}) und messen den
	%TODO Grad (°)?
	%TODO Lot eigentlich in Rot
	jeweiligen Brechungswinkel.
\paragraph{Messung:}
\begin{tabular}{r|r|r|r|r|r|r|r|r|r}
Einfallswinkel \(\alpha\) in \(^\circ\) &
0 & 10 & 20 & 30 & 40 & 50 & 60 & 70 & 80\\
Brechungswinkel \(\beta\) in \(^\circ\) &
0 &  7 & 14 & 20 & 27 & 32 & 37 & 41 & 43
\end{tabular}
\paragraph{Auswertung:} Diagramm: \(\beta(\alpha)\) %TODO
\paragraph{Erklärung/Zusammenhang zwischen \(\beta\) und \(\alpha\)} %TODO \\
Idee: \emph{Huygens-Prinzip} %TODO Farbe: Orange
\begin{itemize}
\item[Dreieck 1:] \(\sin(\alpha) = \frac{c_L \cdot \Delta t}{h}\)
\item[Dreieck 2:] \(\sin(\beta) = \frac{c_G \cdot \Delta t}{h}\)
\end{itemize}
\(\frac{c_L}{c_G} = \frac{\sin(\alpha)}{\sin(\beta)} = n = \mathrm{konstant}\)
\emph{"`relativer"' Brechungsindex}

Jedes Medium hat im Vergleich \underline{zum Vakuum} einen relativen
Brechungsindex, dieser wird als \emph{absoluter Brechungsindex} dieses Mediums
bezeichnet.\\
Beispiele: \(n_\mathrm{Wasser} \approx 1,3;\, n_\mathrm{Glas} \approx 1,5;\,
n_\mathrm{Diamant} \approx 2,4\)\\
Interpretation: Ist der relative Brechungsindex zweier Stoffe groß, so ist die
Brechung besonders stark, d.h. der \emph{Ablenkungswinkel} besonders groß
(\emph{Ablenkungswinkel} \(\delta=\alpha-\beta\))

\subsection{Interferenz am Doppel- und Einzelspalt}
Ein Spalt ist eine schmale Öffnung mit parallelen Rändern, durch die Licht
hindurch fallen kann. Ist der Spalt schmal genug, so ergeben sich
charakteristische \emph{Interferenzmuster}.
%TODO Bild Einzel- und Doppelspalt ([ | ], [ || ])
Da jeder Doppelspalt aus zwei Einzelspalten besteht, überlagern sich hier die
jeweiligen Effekte.
\end{document}
