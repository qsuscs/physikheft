% !TeX spellcheck = de_DE
% !TeX encoding = UTF-8
\documentclass[a4paper]{scrartcl}

\usepackage[utf8]{inputenc}
\usepackage[ngerman]{babel}
\usepackage[T1]{fontenc}
%\usepackage{amsmath}


\begin{document}


\subsubsection{"`Ausdrücke"' für die magnetische Flussdichte}
$B:=\frac{F}{I \cdot l}$ $\leftarrow$ Definition, also eine willkürliche
(aber natürlich sinnvolle) Festlegung\\ %TODO
$B=\frac{F}{q \cdot v}$ aus der Definition abgeleitet für bewegte Ladungen\\
$B=\frac{E}{v}$ aus der Gleichgewichtsbedingung el. und magn. Kraft
($F_{el}=F_{magn}$ $\Longleftrightarrow$ $qE=q \cdot v \cdot B$)\\
$B=\mu_0 \cdot \frac{I}{2 \pi r}$ Flussdichte eines geraden stromdurchflossenen
Leiters\\
$B=\mu_0 \cdot I \cdot \frac{n}{l}$ Flussdichte im innern einer
"`langen"' Spule

\subsubsection{Magnetfeld einer Spule}
\$BILD\\
$l$ = Spulenlänge\\ $n$ = Windungszahl\\ $I$ = Spulenstrom\\
Innerhalb der Spule ist das Magnetfeld näherungsweise homogen, außerhalb
näherungsweise Null, wenn es sich um eine "`\emph{lange Spule}"' handelt, d.h.
die Länge viel größer als ihr Durchmesser ist.

Im Prinzip lässt sich der Ausdruck für B aus dem für den geraden Leiter
herleiten (s. Buch S. 248f). Es gilt $B=\mu_0 I \frac{n}{e}$. $\frac{n}{e}$ kann
als Windungsdichte interpretiert werden.

\emph{Beispiel:} $l=40\mathrm{cm}=0,4\mathrm{m}$; $n=30$; $I=2\mathrm{A}$\\
$B=\mu_0 \cdot I \cdot \frac{n}{e}=4 \pi \cdot 10^{-7}
\frac{\mathrm{Vs}}{\mathrm{Am}}
\cdot 2\mathrm{A} \cdot \frac{30}{0,4\mathrm{m}}=\frac{4 \pi \cdot 30 \cdot
2}{0,4} \cdot 10^{-7} \frac{\mathrm{Vs}}{\mathrm{Am}}= 1900 \cdot 10^{-7} T =
1,9
\cdot 10^{-4} \mathrm{T} = 0,19 \mathrm{mT}$

\subsubsection{Induktion (qualitativ)}
\emph{Induktion} bedeutet, dass mit Hilfe eines Magnetfeldes in einem Leiter
ein Strom erzeugt wird bzw. zwischen den Leiterenden eine Spannung messbar wird.

\paragraph{Mögliche Situationen}
\begin{itemize}
	\item Der Leiter bzw. die Spule wird quer zum Magnetfeld bewegt (bei Spulen
	insbesondere auch Rotation im Feld $\rightarrow$ Generator). Die vom Leiter
	mitgeführten  Elektronen erfahren eine Lorentzkraft \emph{\underline{in
	Richtung}} des Leiters $\Rightarrow$ Strom.
	\item Innerhalb einer Spule (bzw. in der Nähe eines Leiters)
	\underline{ändert}
sich das Magnetfeld. Hier kann die Induktion \emph{nicht} mit der Lorentzkraft
erklärt werden.
\end{itemize}

\subsubsection{Induktion (quantitativ)}
Versuch/Idee: In eine stromdurchflossene Spule (\emph{Feldspule}) wird eine
kleine Spule (\emph{Induktionsspule}) gebracht.
\$BILD
Die Feldspule ist an einen Sägezahngenerator angeschlossen, der die Stromstärke
linear an-- und absteigen lässt, wodurch die magnetische Flussdichte in der
Spule entsprechend an-- und absteigt: $B = \mu_0 \cdot I \cdot \frac{n}{l}$
\$BILD ($B$-$t$-Diagramm der Sägezahnspannung)
Die Induktionsspannung hängt ab von:
\begin{itemize}
	%\item Änderungsrate von $B(t) \rightarrow \dot{B}(t)$
	\item Induktionsspannung hängt ab von $\dot{B}(t)$ (Magnetfeldänderung)
		($\dot{B}(t)$ hängt ab von $I_{max}$ und $f$)
	\item $U_{ind}$ hängt ab von der Querschnittsfläche $A$ der Ind.--Spule
	\item $U_{ind}$ hängt ab von der Windungszahl der Ind.--Spule
\end{itemize}

\paragraph{Messungen}
$n_{ind} = 1000$; %}
%$A_{ind} = 28\mathrm{cm^2}$; %}
$I_{max} = 1\mathrm{A}$; %} U_{ind} = 0,01V
$f=0.15\mathrm{Hz}$ %}
$n_{Feld} = 665$ %} % Versuch 2: f=1Hz %U_ind=0,02V
$\Rightarrow \dot{B}~U_{ind}$ %TODO Tilde
%TODO 3 wie 1, aber n_ind=2000 → U_ind=0.02V → U_ind~n
%TODO 4 wie 3, A_ind=14cm^2, U_ind = 0,01V U_ind ~ A
Insgesamt ergibt sich also $U_ind ~ n \cdot A \cdot \dot{B}$. Wie groß ist der
Proportionalitätsfaktor?
\begin{enumerate}
	\item Einheit des Prop.--faktors: $[n]=1$ $[A]=\mathrm{cm^2}$
	$[B]=\mathrm{T}=
		\frac{\mathrm{Vs}}{\mathrm{m^2}}$
		$[\dot{B}]=\frac{\mathrm{V}}{\mathrm{m^2}}$
		
		$\Rightarrow [n \cdot A \cdot \dot{B}]=V=[U_{ind}]$
		selbe Einheit, d.h. der Prop.--faktor ist dimensionslos (d.h. hat keine
		Einheit)
	\item Wir zeigen anhand der Messwerte, dass der Prop.--faktor 1 ist,
		dass also gilt: $U_{ind} = n \cdot A \cdot \dot{B}$
		%TODO n = 1000, A=28cm^2=0,0028m^2, I_max=1, B_max=\mu_0*I_max*n/l =
		%4\pi *10^-7 Vs/Am *1A 665/0,5m, l=0,5mA, B'=B_max/0,5s
		%TODO U_ind = n_ind*A_ind*B'_Feld =
		%1000*0,0028m^2*(4pi*10^-7Vs/Am*1A*665)/0,5m*0,5s
		%TODO = 2,8*4pi*665/0,5^2 * 10^3*10^-3*10^-7 * VsAm^2/ms
\end{enumerate}

In weiteren Versuchen stellt man fest, dass Verallgemeinert ausgedrückt die
\emph{Änderung} der von dem magnetischen Fluss durchsetzten Spulenfläche
eine Induktionsspannung bewirkt. Dies kann hervorgerufen werden durch
\begin{enumerate}
	\item Zusammendrücken/Auseinanderziehen der Spule im Feld \$BILD %TODO
	\item Herausziehen/Hineinschieben d. Sp. aus dem/in das Feld \$BILD %TODO
	\item Rotation der Spule im Feld
\end{enumerate}
Es gilt hier: $U_{ind} = n_{ind} \cdot \dot{A}_{ind} \cdot B_{Feld}$,
wobei $[\dot{A}]=\frac{\mathrm{m^2}}{\mathrm{s}}$ die
"`\emph{Flächenänderungsrate}"'. %TODO

Wir legen fest: Der \emph{magnetische Fluss} durch eine Fläche ist (bei
homogenem Feld) das Produkt aus mgn. Flussdichte mal betrachteter Fläche:
$\Phi = A \cdot B$ (so ist die Flussdichte $B=\frac{\Phi}{A}$) %TODO
%Fluss frac Fläche
Damitlässt sich das Induktionsgesetz ganz kompakt schreiben:
$U_{ind} = n \cdot \dot{\Phi}$ (Induktionsspannung ist n mal Änderungsrate
des magnetischen Flusses)

($\dot{\Phi}=\dot{A \cdot B} = \dot{A} \cdot B + A \cdot \dot{B}$
(Produktregel))

\subsubsection{Differentialgleichung}
Fall: Entladung/Ausschalten; Schaltung: $C$ oder $L$ parallel zu $R$
geschalten, also $U_C$ bzw. $U_L$ und $U_R$
\\ \begin{tabular}{rl|l}
& Kondensator						& Spule \\ %TODO mathcal==Farbe
"`Konstanten"'&$R$; $C$				& $R$; $L$ \\
Zeitabhängige Größen & $Q(t)=\mathcal{Q}$; $U(t)=\mathcal{U}$;
	$I(t)=\mathcal{I}$ 				& $U(t)=\mathcal{U}$; $I(t)=\mathcal{I}$ \\
& $U_C + U_R = 0$ & $U_L + U_R = 0$ \\
& $\frac{1}{C} \mathcal{Q} + R \cdot \mathcal{I} = 0$ &
	$L \cdot \dot{\mathcal{I}} + R \cdot \mathcal{I}$ = 0 \\
& $\frac{1}{C} \mathcal{Q} + R \dot{\mathcal{Q}} = 0$ &
	$\dot{\mathcal{I}} + \frac{R}{L} \cdot \mathcal{I} = 0$ \\
& $\frac{1}{RC} \mathcal{Q} + \dot{\mathcal{Q}} = 0$ &
	$\dot{\mathcal{I}} = -\frac{R}{L} \cdot \mathcal{I}$ \\
& $\dot{\mathcal{Q}} = - \frac{1}{RC} \mathcal{Q}$ & \\
\hline % Einmal quer drüber "`Lösung der DGL"'
& $\mathcal{Q} = Q(t) = Q_0 \cdot e^{-\frac{1}{RC} \cdot t}$ &
	$\mathcal{I} = I(t) = I_0 \cdot e^{-\frac{R}{L} \cdot t}$ \\
& entsprechend auch für $U(t)$ und $I(t)$ & entsprechend auch für $U(t)$ \\
\emph{"'Probe"':} & $\dot{Q} = Q_0 \cdot e^{-\frac{1}{RC} \cdot t}$ &
	$\dot{I}(t) = I_0 \cdot e^{\frac{R}{L} \cdot t} \cdot -\frac{R}{L}$ \\
& Q/I/U-t-Diagramm & I/U-t-Diagramm \\
\end{tabular}
\paragraph{Einheitenbetrachtung:} $[\frac{1}{RC}] = \mathrm{\frac{1}{\Omega
\cdot F} = \frac{1}{\frac{V}{A} \cdot
\frac{C}{V}} = \frac{A}{C} = \frac{A}{As} = \frac{1}{s}}$;
$[\frac{R}{L}] = \mathrm{\frac{\Omega}{H} = \frac{\frac{V}{A}}{\frac{Vs}{Am}
\cdot \frac{m^2}{m}} = \frac{V \cdot Am \cdot m}{Vs \cdot A \cdot m^2} =
\frac{1}{s}}$


\section{Mechanik -- Schwingungen und Wellen}
\subsection{Die Schwingung einer Schraubenfeder}
\paragraph{Physikalische Größen:} \begin{itemize}
	\item \emph{Frequenz} $f$ $\mathrm{Hz (Hertz)} = \frac{1}{\mathrm{s}}$
	\item \emph{Schwingungsdauer} $T$ $\mathrm{s}$ $f = \frac{1}{T}$
	\item \emph{Amplitude} $\hat{y}$ $\mathrm{m}$
	\item \emph{"`Kreisfrequenz"'} $\omega = 2 \pi f$ $\frac{1}{\mathrm{s}}$
	\item \emph{Auslenkung} $y(t)$ $\mathrm{m}$
	\item \emph{Federkonstante} $D$ $\frac{\mathrm{N}}{\mathrm{m}}$
	\item \emph{Masse} $m$ $\mathrm{kg}$
\end{itemize}
%TODO Bild: Oben "`Decke"', daran befestigt eine Feder, darauf ein grüner Pfeil
%F_D = D*s (Dehnung der Feder), an der Feder eine Masse, davon nach unten ein
%roter Pfeil F_G = m*g, neben der Masse in orange "`Gleichgewichts- oder
%Ruhelage"'
In der Gleichgewichtslage gilt Kräftegleichheit: $F_D = F_G \rightarrow m \cdot
g = D \cdot s_0$

\subsubsection{Beschreibung der Schwingung durch eine Differentialgleichung}
Auftretende Kräfte: Außerhalb der Gleichgewichtslage $s_0$ ist die Federkraft
$F_D$ größer oder kleiner als $F_G = m \cdot g$, die konstant ist. Diese
resultierende Kraft ist die sogen. \emph{rücktreibende Kraft}, die immer in
richtung Ggl. zeigt. Es gilt: $F_R = F_D + F_G = -Ds + mg = -D(y+s_0) + mg =
-Dy-Ds_0 + mg = -Dy$ $F_R = -Dy$ %TODO Kasten rum

$F_R$ ist proportional zur Auslenkung aus der Gleichgewichtslage $y=0$!
(\emph{lineares Kraftgesetz})\\
$F_R$ ist damit die die Masse $m$ beschleunigende Kraft $F=m \cdot a$. Also: $F
= F_R \rightarrow m \cdot a(t) = -D \cdot y(t)$. Da $\dot{y}(t) = v(t)$ und
$\dot{v}(t) = a(t)$ gilt $m \cdot \ddot{y}(t) = -D \cdot y(t)$. Geteilt durch
$m$ ergibt sich $\ddot{y}(t) = -\frac{D}{m} \cdot y(t)$

Diese DGL hat die Struktur $f''(x) = -k \cdot f(x)$. Diese wird gelöst durch
die trigonometrischen Funktionen $\sin(x)$/$\cos(x)$ bzw. genauer $f(x) =
\sin(\sqrt{k}x)$. Denn $f'(x) = \sqrt{k} \cos(\sqrt{k}x)$ und $f''(x) =
-\sqrt{k}^2 \sin(\sqrt{k}x) = -k \sin(\sqrt{k}x)$

Lösung: $y(t) = \hat{y} \cdot \sin(\sqrt{\frac{D}{m}}t)$ ($[\hat{y}] =
\mathrm{m}$; $[\sqrt{\frac{D}{m}}][t] = \mathrm{s} \rightarrow
[\sqrt{\frac{D}{m}}] = \frac{1}{\mathrm{s}}$
\\
$[\frac{D}{m}] = \frac{\mathrm{N}}{\mathrm{m \cdot kg}} =
\frac{1}{\mathrm{s^2}} \rightarrow [\sqrt{\frac{D}{m}}] = \frac{1}{\mathrm{s}}$
\\
$v(t) = \dot{y}(t) = \sqrt{\frac{D}{m}} \cdot \hat{y} \cdot
\cos(\sqrt{\frac{D}{m}}t)$; $\sqrt{\frac{D}{m}} \cdot \hat{y} = \hat{v}$
\\
$a(t) = \ddot{y}(t) = -\sqrt{\frac{D}{m}}^2 \cdot \hat{y} \cdot
\sin(\frac{D}{m} \cdot t)$; $-\sqrt{\frac{D}{m}}^2 \cdot \hat{y} = \hat{a}$,
$\hat{y} \cdot \sin(\frac{D}{m} \cdot t) = y(t)$

Zusammenhang zwischen $y(t) = \hat{y} \sin(\sqrt{\frac{D}{m}} t)$ mit $f$ bzw.
$T$ der Schwingung: Periodendauer/länge: $\sqrt{\frac{D}{m}} \cdot T = 2 \pi$,
d.h. $T=\frac{2 \pi}{\sqrt{\frac{D}{m}}} = 2 \pi \sqrt{\frac{m}{D}}$ $f=2 \pi f
= \sqrt{\frac{D}{m}} = \omega$

\end{document}
