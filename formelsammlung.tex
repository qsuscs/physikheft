\documentclass[paper = a4, twocolumn]{scrartcl}

\usepackage{fontspec}
\defaultfontfeatures{Mapping = tex-text, Ligatures = TeX}
\setmainfont{CMU Serif}
\setsansfont{CMU Sans Serif}
\setmonofont{CMU Typewriter Text}

\usepackage{amsfonts}

\usepackage{polyglossia}
\setmainlanguage[spelling = new, babelshorthands = true]{german}

\usepackage{geometry}
\geometry{top = 1.9cm, left = 1.9cm, right = 1.9cm}

\usepackage{hyperref}

\usepackage{fixltx2e}

\DeclareRobustCommand{\E}{\mathcal{E}}

\title{Formelsammlung Physik}
\author{T.~Schneider, R.~Hofmann, M.~Wolfart, D.~Muffler}
\date{Abitur 2015}

\begin{document}
\maketitle
\tableofcontents

\section{Elektrisches Feld}
\begin{itemize}
\item
	\( E = \frac{F}{q} \)
\item
	\( \Delta Q = \frac{I}{\Delta t} \)
\item
	\( \dot{Q} = I \)
\item
	\( U = \frac{\Delta E}{\Delta Q} \)
\item
	\( U = \Delta \Phi \)
\item
	\( R = \frac{U}{I} \)
\item
	\( \Sigma = \frac{Q}{A} = \epsilon_0 \cdot E \) (Flächenladungsdichte)
\item
	\( E = \frac{q}{4 \cdot \pi \cdot \epsilon_0} \cdot \frac{Q}{r^2} \)
	(Feldstärke einer geladenen Kugel)
\item
	\( F = \frac{1}{4 \cdot \pi \cdot \epsilon_0} \cdot \frac{Q_1 Q_2}{r^2} \)
\item
	\( P = U \cdot I \)
\item
	\( \E = P \cdot t \Rightarrow P = \frac{\Delta \E}{\Delta t} \)
\item
	\( \E = eU \)
\item
	\( \frac{e}{m_e} \) (spezifische Ladung)
\end{itemize}

\subsection{Kondensator}
\begin{itemize}
\item
	\( C = \frac{Q}{U} \)
\item
	\( C = \epsilon_0 \epsilon_r \cdot \frac{A}{d} \)
\item
	\( \E = \frac{1}{2} C U^2 \)
\item
	\( p_e = \frac{1}{2} \epsilon_0 E^2 \) (Energiedichte)
\item
	\( p_e = \frac{\E}{V} \)
\item
	\( \E = F \cdot d = q \cdot E \cdot d \)
\item
	\( E = \frac{U}{d} \)
\item
	\( \Phi = E \cdot d \)
\item
	\( \Sigma = \epsilon_0 \cdot E \)
\item
	\( I(t) = \frac{U}{R} = \frac{U_0 - L \cdot \dot{I}(t)}{R} \)
\end{itemize}

\subsection{Braun’sche Röhre}
\begin{itemize}
\item
	\( \frac{v_y}{v_x} = \frac{y_1 - y_s}{B} \)
\item
	\( v = \sqrt{2 \cdot \frac{E}{m} \cdot U_a} \)
\item
	\( y_1 = \frac{1}{2} \cdot \frac{U}{d} \cdot \frac{1}{U_a} \cdot s \cdot (b
	+ \frac{s}{2} ) \)
\end{itemize}

\subsection{Differentialgleichungen}
\subsubsection{Entladung}
\begin{itemize}
\item
	\( U_C + U_R - |U_0| = 0 \)
\item
	\( U_0 = 0 \)
\item
	\( \frac{Q(t)}{C} + I(t) \cdot R = 0 \)
\item
	\( I(t) = \dot{Q}(t) \)
\item
	\( \dot{Q}(t) = -\frac{1}{RC} \cdot Q(t) \)
\item[\(\Rightarrow\)]
	\( Q(t) = Q_0 \cdot e^{-\frac{1}{RC}} \)
\item
	\( I(t) = I_0 \cdot e^{-\frac{1}{RC} \cdot t} \)
\item
	\( I_0 = \frac{U_0}{R} \)
\item
	\( U(t) = -U_0 \cdot e^{-\frac{1}{RC}} \)
\item
	\( U_0 = R \cdot I(0) \)
\end{itemize}

\subsubsection{Aufladung}
\begin{itemize}
\item
	\( Q(t) = -Q_0 \cdot (1 - e^{-\frac{1}{RC}}) \)
\item
	\( U(t) = -U_0 \cdot (1 - e^{-\frac{1}{RC}}) \)
\item
	\( I(t) = I_0 \cdot  e^{-\frac{1}{RC}} \)
\item
	\( I_0 = \frac{U_0}{R} \)
\end{itemize}

\subsection{Schwingkreis}
\( f_0 = \frac{1}{2 \pi \sqrt{LC}} \) (Thomson’sche Schwingkreisformel)

\section{Magnetisches Feld}
\begin{itemize}
\item
	\( B = \frac{F}{I \cdot l} = \frac{F}{Q \cdot v} \)
\item
	Gerader stromdurchflossener Leiter:\\
	\( B = \mu_0 \cdot \frac{I}{2 \pi r} \)
\item
	\( B = \mu_0 \mu_R \cdot I \cdot \frac{n}{l} \)
\item
	\( \left( H = \frac{F}{q_{mag}}\ \mathrm{(magnetische\ Feldstärke)} \right)
	\)
\item
	\( F_L = I \cdot l \cdot B \cdot \sin \alpha = n \cdot q \cdot v \cdot B
	\cdot \sin \alpha \)
\end{itemize}

\subsection{Индукция -- Induktion}
\begin{itemize}
\item
	\( U_{ind} = -n \cdot \dot{\Phi} \)
\item
	\( \dot{\Phi} = \dot{A} \cdot B + A \cdot \dot{B}; \quad \dot{A} = d \cdot
	v(t) = d \cdot \frac{s}{t} \)
\item
	\(\left(  U_{ind} = b \cdot B \cdot v \right)\)
\item
	\( \dot{B} = \mu_0 \mu_R \cdot \dot{I} \cdot \frac{n}{l} \)
\item
	\( \E_{mag} = \frac{1}{2} \cdot L \cdot I^2 \)
\item
	\( p_{mag} = \frac{1}{2} \cdot \mu_0 \mu_R \cdot B^2 \)
\item
	\( U = \hat{U} \cdot \sin(\omega t) \) (Erzeugung von Wechselspannung)
\item
	\( \hat{U} = n \cdot B \cdot A \cdot \omega \)
\end{itemize}

\subsubsection{Selbstinduktion}
\begin{itemize}
\item
	\( U_{ind} = -L \cdot \dot{I} = -U_L \)
\item
	\( L = \mu_0 \mu_r \cdot \frac{n^2}{l} \cdot A \)
\end{itemize}

\subsection{Entladung der Spule}
\begin{itemize}
\item
	\( I(t) = -I_0 \cdot e^{-\frac{R}{L} \cdot t} \)
\item
	\( U_L(t) = U_0 \cdot e^{-\frac{R}{L} \cdot t} \)
\end{itemize}

\subsection{Aufladung der Spule}
\begin{itemize}
\item
	\( I(t) = -I_0 \cdot (1 - e^{-\frac{R}{L} \cdot t} ) \)
\item
	\( U_L(t) = -U_0 \cdot e^{-\frac{R}{L} \cdot t} \)
\item
	\( I_0 = \frac{U_0}{R} \)
\end{itemize}

\section{Mechanische Schwingungen und Wellen}
\begin{itemize}
\item
	\( \E_{kin} = \frac{1}{2} m v^2 \)
\item
	\( \E_{pot} = m \cdot g \cdot h \)
\item
	\( F = m \cdot a \)
\item
	\( F_G = m \cdot g \)
\item
	\( \phi_{grav} = \frac{\E_{pot}}{m} = g \cdot h \)
\item
	\( p = m \cdot v \)
\item
	\( \omega = 2 \pi f = \frac{v}{r} \)
\item
	\( f = T^{-1} \)
\item
	\( s = \frac{1}{2} \cdot a \cdot t^2 \)
\item
	\( v = a \cdot t \)
\item
	\( F_D = D \cdot s \)
\end{itemize}

\subsection{Harmonischer Oszillator}
\begin{itemize}
\item
	\( \omega = \sqrt{\frac{D}{m}} \)
\item
	\( \E_{pot} = \frac{1}{2} \cdot D \cdot y^2 \quad (y = s) \)
\item
	\( \E_{kin} = \frac{1}{2} v^2(t) \cdot m = \frac{1}{2} \cdot D \cdot
	\hat{y}^2 - \frac{1}{2} \cdot D \cdot y^2(t) \)
\item
	\( \E_{ges} = \frac{1}{2} \cdot D \cdot \hat{y}^2 \)
\end{itemize}

\subsection{Federpendel}
\begin{itemize}
\item
	\( F_R = -F_D + F_G = -D \cdot y \)
\item
	\( y(t) = \hat{y} \cdot \sin(\omega t) \)
\item
	\( \omega = \sqrt{\frac{D}{m}} \)
\end{itemize}

\subsection{Fadenpendel}
\begin{itemize}
\item
	\( F_R = m \cdot g \cdot \sin \alpha = - \frac{m \cdot g}{l} \cdot x \)\\
	\( F_G = F_R \)
\item
	\( \omega = \sqrt{\frac{g}{l}} \)
\item
	\( x(t) = \hat{x} \cdot \sin(\omega t) \)
\item
	\( \hat{x} = \alpha \cdot l \)
\end{itemize}

\subsection{Wellengleichung}
\begin{itemize}
\item
	\( y(x; t) = \hat{y} \cdot \sin(2 \pi \cdot (\frac{t}{T} -
	\frac{x}{\lambda}) ) \)
\end{itemize}

\subsection{Stehende Welle}
\begin{itemize}
\item
	\( c = \lambda \cdot f \)
\item
	\( f_0 = \frac{c}{k \cdot l}; \quad k \in \mathbb{N}; k > 1 \)
\end{itemize}

\subsection{Schwebung}
\begin{itemize}
\item
	\( \Delta f_s = f_1 - f_2 \) (Schwebungsfrequenzbereich)
\item
	\( f_s = \frac{f_1 + f_2}{2} \) (Frequenz der Schwebung)
\end{itemize}

\section{Elektromagnetische Schwingungen und Wellen}
\subsection{Brechung}
\begin{itemize}
\item
	\( \frac{\sin \alpha}{\sin \beta} = \frac{c_1}{c_2} = n \)
\item
	\( n_{1/2} = \frac{n_2}{n_1} \)
\item
	\( n_2 > n_1 \)
\end{itemize}

\subsubsection{Totalreflexion}
\begin{itemize}
\item
	\( \sin \beta_T = \frac{n_1}{n_2} = \frac{1}{n}; \quad n_2 > n_1 \)
\end{itemize}

\subsection{Verknüpfung elektrischer und magnetischer Wechselfelder}
\begin{itemize}
\item
	\( E = B \cdot c \)
\item
	\( c = \frac{1}{\sqrt{\epsilon_0 \cdot \mu_0}} \)
\end{itemize}

\subsection{Interferenz}
\begin{itemize}
\item
	\( a_k = \frac{\Delta s}{d} \cdot l; \quad \Delta s = k \cdot \lambda \)
	(Maximumsbedingung bei Doppelspalt (\( k \in \mathbb{Z} \)),
	Minimumsbedingung bei Einzelspalt (\( k \in \mathbb{Z}\textbackslash\{0\}
	\))
\item
	\( a_k = \frac{\Delta s}{d} \cdot l; \quad \Delta s = (k + \frac{1}{2})
	\cdot \lambda \) (Minimumsbedingung bei Doppelspalt (\( k \in
	\mathbb{Z} \)), Maximumsbedingung bei Einzelspalt (\( k
	\in \mathbb{Z} \))
\end{itemize}

\subsubsection{Gitter}
\begin{itemize}
\item
	\( \Delta s = \sin(\alpha) \cdot g \quad \Rightarrow \quad k \cdot
	\frac{\lambda}{g} < 1 \)
\item
	\( a_k = l \cdot \tan(\arcsin(k \cdot \frac{\lambda}{g})) \)
\end{itemize}

\subsubsection{Brewster-Winkel}
\( \tan \alpha = n \)

\section{Quantenphysik}

\end{document}
