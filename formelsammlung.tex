\documentclass[paper = a4, twocolumn]{scrartcl}

\usepackage{fontspec}
\defaultfontfeatures{Mapping = tex-text, Ligatures = TeX}
\setmainfont{CMU Serif}
\setsansfont{CMU Sans Serif}
\setmonofont{CMU Typewriter Text}

\usepackage{amsfonts}

\usepackage{polyglossia}
\setmainlanguage[spelling = new, babelshorthands = true]{german}

\usepackage{geometry}
\geometry{top = 1.9cm, left = 1.9cm, right = 1.9cm}

\usepackage{hyperref}

\usepackage{fixltx2e}

\DeclareRobustCommand{\E}{\mathcal{E}}

\title{Formelsammlung Physik}
\author{T.~Schneider, R.~Hofmann, M.~Wolfart, D.~Muffler, N.~Schuller}
\date{Abitur 2015}

\begin{document}
\maketitle
\tableofcontents

\section{Elektrisches Feld}
\begin{itemize}
\item
	\( E = \frac{F}{q} \)
\item
	\( I = \frac{\Delta Q}{\Delta t} \)
\item
	\( I = \dot{Q} \)
\item
	\( U = \frac{\Delta E}{\Delta Q} \)
\item
	\( U = \Delta \varphi \)
\item
	\( R = \frac{U}{I} \)
\item
	\( \sigma = \frac{Q}{A} = \varepsilon_0 \cdot E \)
	(Flächenladungsdichte)
\item
	\( E = \frac{1}{4 \cdot \pi \cdot \varepsilon_0} \cdot \frac{Q}{r^2} \)
	(Feldstärke einer geladenen Kugel)
\item
	\( F = \frac{1}{4 \cdot \pi \cdot \varepsilon_0} \cdot \frac{Q_1 Q_2}{r^2} \)
\item
	\( P = U \cdot I \)
\item
	\( \Delta \E = P \cdot \Delta t \Leftrightarrow P = \frac{\Delta
	\E}{\Delta t} \mbox{ bzw. } P=\dot{\E} \)
\item
	\( \E = qU \)
\item
	\( \frac{e}{m_e} \) (spezifische Ladung)
\end{itemize}

\subsection{Kondensator}
\begin{itemize}
\item
	\( C = \frac{Q}{U} \)
\item
	\( C = \varepsilon_0 \varepsilon_r \cdot \frac{A}{d} \)
\item
	\( \E = \frac{1}{2} C U^2 \)
\item
	\( \rho_e = \frac{1}{2} \varepsilon_0 E^2 \) (Energiedichte)
\item
	\( \rho_e = \frac{\E}{V} \)
\item
	\( \E = F \cdot d = q \cdot E \cdot d \)
\item
	\( E = \frac{U}{d} \)
\item
	\( U = E \cdot d \)
\item
	\( \sigma = \varepsilon_0 \cdot E \)
\item
	\( I(t) = \frac{U}{R} = \frac{U_0 - L \cdot \dot{I}(t)}{R} \)
\end{itemize}

\subsection{Braun’sche Röhre}

nicht merken. Herleitung evtl.\ wissen
\begin{itemize}
\item
	\( \frac{v_y}{v_x} = \frac{y_1 - y_s}{B} \)
\item
	\( v = \sqrt{2 \cdot \frac{E}{m} \cdot U_a} \)
\item
	\( y_1 = \frac{1}{2} \cdot \frac{U}{d} \cdot \frac{1}{U_a} \cdot s \cdot (b
	+ \frac{s}{2} ) \)
\end{itemize}

\subsection{Auf-  und Entladung eines Kondensators}
\subsubsection{Entladung}
\begin{itemize}
\item
	\( U_C + U_R - |U_0| = 0 \)
\item
	\( U_0 = 0 \)
\item
	\( \frac{Q(t)}{C} + I(t) \cdot R = 0 \)
\item
	\( I(t) = \dot{Q}(t) \)
\item
	\( \dot{Q}(t) = -\frac{1}{RC} \cdot Q(t) \)
\item[\(\Rightarrow\)]
	\( Q(t) = Q_0 \cdot e^{-\frac{1}{RC}} \)
\item
	\( I(t) = I_0 \cdot e^{-\frac{1}{RC} \cdot t} \)
\item
	\( I_0 = \frac{U_0}{R} \)
\item
	\( U(t) = -U_0 \cdot e^{-\frac{1}{RC}} \)
\item
	\( U_0 = R \cdot I(0) \)
\end{itemize}

\subsubsection{Aufladung}
\begin{itemize}
\item
	\( Q(t) = -Q_0 \cdot (1 - e^{-\frac{1}{RC}}) \)
\item
	\( U(t) = -U_0 \cdot (1 - e^{-\frac{1}{RC}}) \)
\item
	\( I(t) = I_0 \cdot  e^{-\frac{1}{RC}} \)
\item
	\( I_0 = \frac{U_0}{R} \)
\end{itemize}

\subsection{Schwingkreis}
\( f_0 = \frac{1}{2 \pi \sqrt{LC}} \) (Thomson’sche Schwingkreisformel)

\section{Magnetisches Feld}
\begin{itemize}
\item
	\( B = \frac{F}{I \cdot l} = \frac{F}{Q \cdot v} \)
\item
	Gerader stromdurchflossener Leiter:\\
	\( B = \mu_0 \cdot \frac{I}{2 \pi r} \)
\item
	\( B = \mu_0 \mu_R \cdot I \cdot \frac{n}{l} \)
\item
	\( \left( H = \frac{F}{q_{mag}}\ \mathrm{(magnetische\ Feldstärke)} \right)
	\)
\item
	\( F_L = I \cdot l \cdot B \cdot \sin \alpha  \)
\item
	\( F_L =  q \cdot v \cdot B \cdot \sin \alpha \)
\end{itemize}

\subsection{Induktion}
\begin{itemize}
\item
	\( U_{ind} = -n \cdot \dot{\Phi} \)
\item
	\( \dot{\Phi} = \dot{A} \cdot B + A \cdot \dot{B}; \quad \dot{A} = d \cdot
	v(t) = d \cdot \frac{s}{t} \)
\item
	\(\left(  U_{ind} = b \cdot B \cdot v \right)\)
\item
	\( \dot{B} = \mu_0 \mu_R \cdot \dot{I} \cdot \frac{n}{l} \)
\item
	\( \E_{mag} = \frac{1}{2} \cdot L \cdot I^2 \)
\item
	\( \rho_{mag} = \frac{1}{2} \cdot \mu_0 \mu_R \cdot B^2 \)
\item
	\( U = \hat{U} \cdot \sin(\omega t) \) (Erzeugung von Wechselspannung)
\item
	\( \hat{U} = n \cdot B \cdot A \cdot \omega \)
\end{itemize}

\subsubsection{Selbstinduktion}
\begin{itemize}
\item
	\( U_{ind} = -L \cdot \dot{I} = -U_L \)
\item
	\( L = \mu_0 \mu_r \cdot \frac{n^2}{l} \cdot A \)
\end{itemize}

\subsection{„Abschalten“ der Spule}
\begin{itemize}
\item
	\( I(t) = -I_0 \cdot e^{-\frac{R}{L} \cdot t} \)
\item
	\( U_L(t) = U_0 \cdot e^{-\frac{R}{L} \cdot t} \)
\end{itemize}

\subsection{„Einschalten“ der Spule}
\begin{itemize}
\item
	\( I(t) = -I_0 \cdot (1 - e^{-\frac{R}{L} \cdot t} ) \)
\item
	\( U_L(t) = -U_0 \cdot e^{-\frac{R}{L} \cdot t} \)
\item
	\( I_0 = \frac{U_0}{R} \)
\end{itemize}

\section{Mechanische Schwingungen und Wellen}
\begin{itemize}
\item
	\( v = \frac{\Delta s}{\Delta t} \)
\item
	\( \E_{kin} = \frac{1}{2} m v^2 \)
\item
	\( \E_{pot} = m \cdot g \cdot h \)
\item
	\( F = m \cdot a \)
\item
	\( F_G = m \cdot g \)
\item
	\( \phi_{grav} = \frac{\E_{pot}}{m} = g \cdot h \)
\item
	\( p = m \cdot v \)
\item
	\( \omega = 2 \pi f = \frac{v}{r} \)
\item
	\( f = T^{-1} \)
\item
	\( s = \frac{1}{2} \cdot a \cdot t^2 \)
\item
	\( v = a \cdot t \)
\item
	\( F_D = D \cdot s \)
\end{itemize}

\subsection{Harmonischer Oszillator}
\begin{itemize}
\item
	\( \omega = \sqrt{\frac{D}{m}} \)
\item
	\( \E_{pot} = \frac{1}{2} \cdot D \cdot y^2 \quad (y = s) \)
\item
	\( \E_{kin} = \frac{1}{2} v^2(t) \cdot m = \frac{1}{2} \cdot D \cdot
	\hat{y}^2 - \frac{1}{2} \cdot D \cdot y^2(t) \)
\item
	\( \E_{ges} = \frac{1}{2} \cdot D \cdot \hat{y}^2 \)
\end{itemize}

\subsection{Federpendel}
\begin{itemize}
\item
	\( F_R = -F_D + F_G = -D \cdot y \)
\item
	\( y(t) = \hat{y} \cdot \sin(\omega t) \)
\item
	\( \omega = \sqrt{\frac{D}{m}} \)
\end{itemize}

\subsection{Fadenpendel}
\begin{itemize}
\item
	\( F_R = m \cdot g \cdot \sin \alpha = - \frac{m \cdot g}{l} \cdot x \)\\
	\( F_G = F_R \)
\item
	\( \omega = \sqrt{\frac{g}{l}} \)
\item
	\( x(t) = \hat{x} \cdot \sin(\omega t) \)
\item
	\( \hat{x} = \alpha \cdot l \)
\end{itemize}

\subsection{Wellengleichung}
\begin{itemize}
\item
	\( y(x; t) = \hat{y} \cdot \sin(2 \pi \cdot (\frac{t}{T} -
	\frac{x}{\lambda}) ) \)
\end{itemize}

\subsection{Stehende Welle}
\begin{itemize}
\item
	\( c = \lambda \cdot f \)
\item
	\( \lambda_k = 2l/(k+1) \quad k \in \mathbb{N}\)\\
	(Wellenlängenbedingung bei gleichen Enden)
\item
	\( \lambda_k = 4l/(2k+1) \quad k \in \mathbb{N}\)\\
	(ein freies und ein festes Ende)
\item
	\( f_k = \frac{c}{\lambda_k} \quad k \in \mathbb{N}\)
\end{itemize}

\subsection{Schwebung}
\begin{itemize}
\item
	\( \Delta f_s = f_1 - f_2 \) (Schwebungsfrequenzbereich)
\item
	\( f_s = \frac{f_1 + f_2}{2} \) (Frequenz der Schwebung)
\end{itemize}

\section{Elektromagnetische Schwingungen und Wellen}
\subsection{Brechung}
\begin{itemize}
\item
	\( \frac{\sin \alpha}{\sin \beta} = \frac{c_1}{c_2} = n \)
\item
	\( n_{1/2} = \frac{n_2}{n_1} \)
\item
	\( n_2 > n_1 \)
\end{itemize}

\subsubsection{Totalreflexion}
\begin{itemize}
\item
	\( \sin \beta_T = \frac{n_1}{n_2} = \frac{1}{n}; \quad n_2 > n_1 \)
\end{itemize}

\subsection{Verknüpfung elektrischer und magnetischer Wechselfelder}
\begin{itemize}
\item
	\( E = B \cdot c \)
\item
	\( c = \frac{1}{\sqrt{\epsilon_0 \cdot \mu_0}} \)
\end{itemize}

\subsection{Interferenz}
\begin{itemize}
\item
	\( a_k = \frac{\Delta s}{d} \cdot l; \quad \Delta s = k \cdot \lambda \)
	(Maximumsbedingung bei Doppelspalt (\( k \in \mathbb{Z} \)),
	Minimumsbedingung bei Einzelspalt (\( k \in \mathbb{Z}\setminus\{0\} \))
\item
	\( a_k = \frac{\Delta s}{d} \cdot l; \quad \Delta s = (k + \frac{1}{2})
	\cdot \lambda \) (Minimumsbedingung bei Doppelspalt (\( k \in
	\mathbb{Z} \)), Maximumsbedingung bei Einzelspalt (\( k
	\in \mathbb{Z} \))
\end{itemize}

\subsubsection{Gitter}
\begin{itemize}
\item
	\( \Delta s = \sin(\alpha) \cdot g \quad \Rightarrow \quad k \cdot
	\frac{\lambda}{g} < 1 \)
\item
	\( a_k = l \cdot \tan(\arcsin(k \cdot \frac{\lambda}{g})) \)
\end{itemize}

\subsubsection{Brewster-Winkel}
\( \tan \alpha = n \)

\section{Quantenphysik}
\begin{itemize}
\item
	\( \lambda = \frac{h}{p} \)
\item
	\( p = m \cdot v \)
\item
	\( m = \frac{h}{\lambda \cdot c} = \frac{h \cdot f}{\lambda \cdot f \cdot c}
	= \frac{\E}{c^2} \)
\item[\(\Rightarrow\)]
	\( \E = m \cdot c^2 \)
\item
	\(\Delta f \cdot \Delta t = 1\) (Bandbreite, Frequenzumfang)

\end{itemize}

\subsection{Fotoeffekt}
\begin{itemize}
\item
	\( \E_{Ph} = \E_{FE} + \E_A = h \cdot f \)
\item
	\( \E_{FE} = h \cdot f - \E_A \)

\end{itemize}

\subsection{Heisenberg’sche Unschärferelationen}
\begin{itemize}
\item
	\( \Delta t \cdot \Delta f = \frac{1}{2} \) („Tonunschärfe“)
\item
	\( \Delta t \cdot \Delta \E = \frac{h}{2} \)
\item
	\( \Delta t \cdot \Delta \E \le \frac{h}{4 \cdot \pi} \)
\item
	\( \Delta x \cdot \Delta p_x = \frac{h}{2} \)
\item
	\( \Delta x \cdot \Delta p_x \le \frac{h}{4 \cdot \pi} \)
\end{itemize}

\subsection{Kohärenz}
\begin{itemize}
\item
	\( \Delta x = c \cdot \Delta t \) (Kohärenzlänge)
\end{itemize}

\section{Wärmelehre}
\begin{itemize}
\item
	\( \Delta S = \frac{\Delta \E}{T} \)
\item
	\( P = \Delta T \cdot I_S \)
\item
	\( I_S = \frac{\Delta S}{\Delta t} \)
\end{itemize}

\pagebreak
\begin{appendix}
\section{Lizenz}
\includegraphics[width = 1.5cm]{cc0.png}

Die Autoren (s.\,o.) haben dieses Werk in die Gemeinfreiheit -- auch genannt
Public Domain -- \emph{entlassen}, indem sie weltweit auf alle
urheberrechtlichen und verwandten Schutzrechte verzichtet haben, soweit das
gesetzlich möglich ist.

Sie dürfen das Werk kopieren, verändern, verbreiten und aufführen, sogar zu
kommerziellen Zwecken, ohne um weitere Erlaubnis bitten zu müssen.
\\Siehe \href{https://creativecommons.org/publicdomain/zero/1.0/deed.de}
{\texttt{https://creativecommons.org/\\publicdomain/zero/1.0/deed.de}} für
Details.
\end{appendix}

\end{document}
