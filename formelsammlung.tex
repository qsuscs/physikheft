\documentclass[paper = a4, twocolumn]{scrartcl}

\usepackage{fontspec}
\setmainfont{CMU Serif}
\setsansfont{CMU Sans Serif}
\setmonofont{CMU Typewriter Text}

\usepackage{polyglossia}
\setmainlanguage[spelling = new]{german}

\usepackage{geometry}
\geometry{top = 1.9cm, left = 1.9cm, right = 1.9cm}

\usepackage{hyperref}

\usepackage{fixltx2e}

\DeclareRobustCommand{\E}{\mathcal{E}}

\title{Formelsammlung Physik}
\author{T.~Schneider, R.~Hofmann, M.~Wolfart, D.~Muffler}
\date{Abitur 2015}

\begin{document}
\maketitle
\tableofcontents

\section{Elektrisches Feld}
\begin{itemize}
\item
	\( E = \frac{F}{q} \)
\item
	\( \Delta Q = \frac{I}{\Delta t} \)
\item
	\( \dot{Q} = I \)
\item
	\( U = \frac{\Delta E}{\Delta Q} \)
\item
	\( U = \Delta \Phi \)
\item
	\( R = \frac{U}{I} \)
\item
	\( \Sigma = \frac{Q}{A} = \epsilon_0 \cdot E \) (Flächenladungsdichte)
\item
	\( E = \frac{q}{4 \cdot \pi \cdot \epsilon_0} \cdot \frac{Q}{r^2} \)
	(Feldstärke einer geladenen Kugel)
\item
	\( F = \frac{1}{4 \cdot \pi \cdot \epsilon_0} \cdot \frac{Q_1 Q_2}{r^2} \)
\item
	\( P = U \cdot I \)
\item
	\( \E = P \cdot t \Rightarrow P = \frac{\Delta \E}{\Delta t} \)
\item
	\( \E = eU \)
\item
	\( \frac{e}{m_e} \) (spezifische Ladung)
\end{itemize}

\subsection{Kondensator}
\begin{itemize}
\item
	\( C = \frac{Q}{U} \)
\item
	\( C = \epsilon_0 \epsilon_r \cdot \frac{A}{d} \)
\item
	\( \E = \frac{1}{2} C U^2 \)
\item
	\( p_e = \frac{1}{2} \epsilon_0 E^2 \) (Energiedichte)
\item
	\( p_e = \frac{\E}{V} \)
\item
	\( \E = F \cdot d = q \cdot E \cdot d \)
\item
	\( E = \frac{U}{d} \)
\item
	\( \Phi = E \cdot d \)
\item
	\( \Sigma = \epsilon_0 \cdot E \)
\end{itemize}

\subsection{Braun’sche Röhre}
\begin{itemize}
\item
	\( \frac{v_y}{v_x} = \frac{y_1 - y_s}{B} \)
\item
	\( v = \sqrt{2 \cdot \frac{E}{m} \cdot U_a} \)
\item
	\( y_1 = \frac{1}{2} \cdot \frac{U}{d} \cdot \frac{1}{U_a} \cdot s \cdot (b
	+ \frac{s}{2} ) \)
\end{itemize}

\subsection{Differentialgleichungen}
\subsubsection{Entladung}
\begin{itemize}
\item
	\( U_C + U_R - |U_0| = 0 \)
\item
	\( U_0 = 0 \)
\item
	\( \frac{Q(t)}{C} + I(t) \cdot R = 0 \)
\item
	\( I(t) = \dot{Q}(t) \)
\item
	\( \dot{Q}(t) = -\frac{1}{RC} \cdot Q(t) \)
\item[\(\Rightarrow\)]
	\( Q(t) = Q_0 \cdot e^{-\frac{1}{RC}} \)
\item
	\( I(t) = I_0 \cdot e^{-\frac{1}{RC} \cdot t} \)
\item
	\( I_0 = \frac{U_0}{R} \)
\item
	\( U(t) = -U_0 \cdot e^{-\frac{1}{RC}} \)
\item
	\( U_0 = R \cdot I(0) \)
\end{itemize}

\subsubsection{Aufladung}
\begin{itemize}
\item
	\( Q(t) = -Q_0 \cdot (1 - e^{-\frac{1}{RC}}) \)
\item
	\( U(t) = -U_0 \cdot (1 - e^{-\frac{1}{RC}}) \)
\item
	\( I(t) = I_0 \cdot  e^{-\frac{1}{RC}} \)
\item
	\( I_0 = \frac{U_0}{R} \)
\end{itemize}

\subsection{Schwingkreis}
\( f_0 = \frac{1}{2 \pi \sqrt{LC}} \) (Thomson’sche Schwingkreisformel)

\section{Magnetisches Feld}
ниггэр
\begin{itemize}
\item
	\( B = \frac{F}{I \cdot l} = \frac{F}{Q \cdot v} \)
\item
	Gerader stromdurchflossener Leiter:\\
	\( B = \mu_0 \cdot \frac{I}{2 \pi r} \)
\item
	\( B = \mu_0 \mu_R \cdot I \cdot \frac{n}{l} \)
\item
	\( \left( H = \frac{F}{q_{mag}}\ \mathrm{(magnetische\ Feldstärke)} \right)
	\)
\item
	\( F_L = I \cdot l \cdot B \cdot \sin \alpha = n \cdot q \cdot v \cdot B
	\cdot \sin \alpha \)
\end{itemize}

\subsection{Индукция – Induktion}
\begin{itemize}
\item
	\( U_{ind} = -n \cdot \dot{\Phi} \)
\item
	\( \dot{\Phi} = \dot{A} \cdot B + A \cdot \dot{B} \)
\item
	\(\left(  U_{ind} = b \cdot B \cdot v \right)\)
\item
	\( \dot{B} = \mu_0 \mu_R \cdot \dot{I} \cdot \frac{n}{l} \)
\item
	\( U_{ind} = -L \cdot \dot{I} = -U_L \)
\item
	\( L = \mu_0 \mu_r \cdot \frac{n^2}{l} \cdot A \)
\end{itemize}

\end{document}